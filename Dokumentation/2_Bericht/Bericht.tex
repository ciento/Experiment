% !TEX program = arara
% arara: pdflatex
% arara: biber
% arara: pdflatex
% arara: pdflatex
% arara: clean: { files: [ Bericht.out ] }
% arara: clean: { files: [ Bericht.aux, Bericht.bbl ] }
% arara: clean: { files: [ Bericht.bcf, Bericht.blg ] }
% arara: clean: { files: [ Bericht.log, Bericht.run.xml ] }
% arara: clean: { files: [ Bericht.toc, Bericht-blx.bib ] }
% 

\documentclass{Bericht}
\usepackage{todonotes}
\usepackage[utf8]{inputenc}
\usepackage{hyperref} % http://ctan.org/pkg/hyperref
\hypersetup{
	colorlinks = true,
	linkcolor  = black
}

\begin{document}

\maketitle

% % % % %

\tableofcontents
\clearpage

\section{Einleitung}
	\todo[inline]{Autor: Kim}

\section{Ideenfindung}
	\todo[inline]{Autor: Berna}
	Hier steht die Ideenfindung.

\section{Hypothese}
	\todo[inline]{Autor: Berna}
		Hier steht die Hypothese.

\section{Verlauf}
	\todo[inline]{Autor: Marvin \& Daniel}
	\subsection{Planung/Recherche} % Marvin
	\subsection{Design} % Daniel
		methodisches versuchsdesign
	\subsection{Versuchsdurchführung} % Daniel
	\subsection{Auswertung} % Marvin

\section{Versuchsaufbau und Durchführung}
	\todo[inline]{Autor: Nizan}
	\subsection{Probanden}
	\subsection{Aufbau \& Apparaturen}
	\subsection{Experiment Ablauf}
	\subsubsection{Fragebögen}
	\subsubsection{Intelligenztest}
	\subsubsection{Genaue Wortlaute bei der Testphase}
	\subsubsection{Schwierigkeiten}
	\subsection{Ethik}

\section{Beobachtungen}
	\todo[inline]{Autor: Nicole \& Svenja}
	Nur Verweis auf Paper!

\section{Auswertung}
	\todo[inline]{Autor: Nicole \& Svenja}
	Nur Verweis auf Paper!

\section{Interpretation}
	\todo[inline]{Autor: Nicole \& Svenja}
	Nur Verweis auf Paper!
	
\section{Fazit}
	\todo[inline]{Autor: Chovi}

	Zusammenfassung: Was wurde grob in dieser Arbeit gemacht? Was waren die Schlüsselergebnisse?

	Wem nützen die Beiträge der Arbeit? Inwieweit wird die Wissenschaft durch die Arbeit verbessert?

	Aufbauende Arbeiten: Welche möglichen Lösungsansätze für noch bestehende Probleme sind denkbar(mehr Probanden)? Wie könnten Folgearbeiten aussehen?

\section{Abschließende Gedanken}
	\todo[inline]{Autor: Jana}
	Wir sind in den verschiedenen Phasen des Projekts auf unterschiedliche Probleme gestoßen, von denen wir viele zeitnah lösen konnten. Allerdings gab es auch schwerwiegendere, deren Lösung für uns nicht gleich oder gar nicht offensichtlich war und die unseren ursprünglichen Zeitplan infolgedessen ungewollt stark beeinflusst haben.\\
	Eines der größten Hindernisse in unserem Projekt waren technische Probleme, die ein Vorankommen sehr stark verzögerten und in einem veränderten Zeitplan resultierten.\\
	Durch die Inkompabilität der Ocolus Rift und dem Computer der Virtusphere konnten wir unser Projekt nicht wie ursprünglich geplant in der Virtusphere stattfinden lassen und mussten auf herkömmliche Controller ausweichen. Diese Umstellung ist sehr ärgerlich, da viel Zeit in die Problemlösung investiert wurde und wir den Probanden letztendlich doch nicht das bieten konnten, mit dem wir sie ursprünglich geworben haben. 
	Weiterhin zu kritisieren wäre eine nicht ganz ideale Arbeitsteilung, wodurch es gerade am Ende zu erheblichem Zeitverzug kam. Insbesondere das sog. \textit{Technikteam}, welches primär für die Erstellung der Blenderobjekte, das Einsetzen und das Animieren der Welt verantwortlich war, hat gleich zu Beginn des Projekts anstehende Aufgaben nicht sinnvoll verteilt. Alle Mitglieder waren gleichermaßen nur mit \textit{aktuellen} Aufgaben beschäftigt. D.h., dass zunächst alle für die Erstellung von Blenderobjekten verantwortlich waren und so Kapazitäten und Zeit verschwendet wurde, da sich bisher noch keiner besonders gut mit der Software auskannte und sich alle gleichermaßen neu einarbeiten mussten.\\
	Diese Entscheidung mag daraus resultieren, dass wir im Bezug auf so eine Art von selbstorganisierter Arbeit sehr unerfahren sind und es dadurch zu Fehleinschätzungen kam.\\
	Im Nachhinein stellen wir weiterhin fest, dass es für das Projekt besser gewesen wäre, hätten wir zu Beginn andere Prioritäten festgelegt. Da die Logik der Welt der Punkt ist, von dem die Ergebnisse abhängen, hätte man sich zuallererst auch mit dieser beschäftigen müssen bzw. rechtzeitig die Personen bestimmen müssen, die die Hauptverantwortung für diese übernehmen.\\
	Auch diese Fehleinschätzung zeigte sich im Nachhinein als kritisch.

\section{Appendix} % = Anhang
	\todo[inline]{Autor: Alina}
	Anhang.
	
\vfill %Zum Seitenende Verschieben

\printbibliography

\end{document}
